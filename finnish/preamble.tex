% "When the new output routine for LaTeX3 is done, this package will be obsolete. The sooner the better..."
\RequirePackage{dblfloatfix} % älä lataa: stfloats, fix2col; sisältää: fixltx2e
\RequirePackage{fix-cm} % ei tarvita: type1cm, type1ec

\documentclass[a4paper,onecolumn,11pt,finnish,twoside,final,oldtoc]{boek3}
\usepackage{calc} % geometry-paketti hyötyy tästä
\usepackage[papersize={185mm,240mm},
            left=26mm,right=22mm,
            top=15mm,bottom=20mm,%
            asymmetric,bindingoffset=10mm,marginparwidth=30mm]{geometry}
\reversemarginpar % marginaalit toisin päin
\setcounter{secnumdepth}{7} % kaikki numeroidaan ilman *-optiota
\usepackage{xifthen} % if-else

% -----

% LuaLaTeX
\usepackage[normalem]{ulem} % alleviivaus ym.
\usepackage{texnames} % erikoismerkit ym.
\usepackage{mathtools}
\usepackage{amssymb}
\usepackage{amsthm}
\usepackage{cmap}
\usepackage{fontspec}
\usepackage{unicode-math}
%\usepackage{upgreek}
\defaultfontfeatures{Ligatures=TeX}
\setmainfont[Extension=.otf,UprightFont=*-regular,ItalicFont=*-italic,BoldFont=*-bold,BoldItalicFont=*-bolditalic]{texgyretermes}
\setsansfont[Extension=.otf,UprightFont=*-regular,ItalicFont=*-italic,BoldFont=*-bold,BoldItalicFont=*-bolditalic]{texgyreadventor}
\setmathfont{xits-math.otf}
\newfontface\fontII[Scale=2.3]{texgyreadventor-regular.otf}
\newfontface\fontIII[Scale=1.8]{texgyreadventor-regular.otf}
\newfontface\fontIV[Scale=0.8]{texgyreadventor-regular.otf}
\newfontface\fontV[Scale=1.4]{texgyreadventor-regular.otf}
\newfontface\fontVI[Scale=1.5]{texgyreadventor-regular.otf}
\usepackage[protrusion=true,expansion=true,verbose=true]{microtype}

% -----

\newcommand{\kurssi}{}
\newcommand{\painos}{}
\newcommand{\ISBN}{}
\newcommand{\painopaikka}{}
\input{config/meta}

% -----

\raggedbottom

% Otsikoiden säätelyyn
\usepackage{titlesec}

\makeatletter
\g@addto@macro\verbatim{\pdfprotrudechars=0 \pdfadjustspacing=0\relax}
\makeatother

\usepackage[finnish,english]{babel}
\usepackage{everyshi} % totpages vaatii
\usepackage{setspace} % rivivälin säätöön
\onehalfspacing % riviväli 1,5
\usepackage[obeyspaces,spaces,hyphens,T1]{url} % URL:ien ladonta ja "tavutus"

\usepackage{pgf,tikz}
\usetikzlibrary{arrows}

% Tunnisteiden ja sivunumeroiden asettamista varten
\usepackage{fancyhdr}

% VerbatimOut Pythonia varten
\usepackage{fancyvrb}
\usepackage{fancybox}

\usepackage{eurosym} % euron merkki
\usepackage{wasysym} % promillemerkki ym.
\usepackage{answers} % harjoitustehtäviä varten
\usepackage{cancel} % supistaminen ym.

% Monipalstainen sisältö esim. lyhyille tehtäville
% Käyttö: \begin{multicols}{n}
%              ...
%
% Älkää käyttäkö, ei se kuitenkaan toimi oikein. --NVI         
%\usepackage{multicol}

\setlength{\columnsep}{0.15cm}

\usepackage{easylist} % edistynyt listaympäristö, ks. alakohdat.sty

% Desimaalipilkut
% Käyttö: väli pilkun jälkeen -> listapilkku; ei väliä pilkun jälkeen -> desimaalipilkku
\usepackage{icomma}

% Automatisoitu hakemisto
\usepackage{makeidx}
\makeindex

\usepackage{datetime} % kustomoidut aika- ja päivämääräformaatit
\usepackage{scrextend} % komento \ifthispageodd

% Marginaalien muuttaminen harjoitustehtäväsivuja varten
\usepackage[strict]{changepage}

\usepackage{rotating} % käännetyt tekstit taulukoita varten
\usepackage{array} % taulukot

%% Testailuun, komentoja: \blindtext, \Blindtext, \pagevalues
%\usepackage{blindtext}
%\usepackage{layouts}

% Ei sisennetä kappaleen ekaa riviä
\setlength{\parindent}{0.0cm}

\usepackage{graphicx} % lisäattribuutteja \includegraphics-komentoon

\input{longdiv} % long division (huom! longdiv.tex, siksi input)

% Hypersetup (metatason pdf-säätöä)
\input{config/hypersetup}

% Omat komennot
\input{commons/finnish/constants}

\newcommand{\vari}{}
\definecolor{vapaa_matikka_vari_3}{cmyk}{0.20,0.27,0.00,0.00}   % Laatikot
\definecolor{vapaa_matikka_vari_4}{cmyk}{1.00,0.42,0.65,0.00}   % Esimerkit
\definecolor{vapaa_matikka_vari_5}{cmyk}{0.60,0.80,0.00,0.00}   % Tehtävänumerot

% Katso huomautus ylempänä.
%\ifgrey
%    \newcommand{\vari}{_grey}
%    \definecolor{vapaa_matikka_vari_3}{cmyk}{0.00,0.00,0.00,0.25}   % Laatikot
%    \definecolor{vapaa_matikka_vari_4}{cmyk}{0.00,0.00,0.00,0.25}   % Esimerkit
%    \definecolor{vapaa_matikka_vari_5}{cmyk}{0.00,0.00,0.00,0.80}   % Tehtävänumerot
%\else
%    \newcommand{\vari}{}
%    \definecolor{vapaa_matikka_vari_3}{cmyk}{0.20,0.27,0.00,0.00}	% Laatikot
%    \definecolor{vapaa_matikka_vari_4}{cmyk}{1.00,0.42,0.65,0.00}	% Esimerkit
%    \definecolor{vapaa_matikka_vari_5}{cmyk}{0.60,0.80,0.00,0.00}	% Tehtävänumerot
%\fi

% Tarvitaan kuvien ja taulukkojen vierekkäin laittamiseen.
\def\vcent#1{\mathsurround0pt$\vcenter{\hbox{#1}}$}

\usepackage{commons/packages/erikoissivut}
\usepackage{commons/packages/esimerkki}
\usepackage{commons/packages/kansilehti}
\usepackage{commons/packages/koosterakenteet}
\usepackage{commons/packages/kuva}
\usepackage{commons/packages/kuvaaja}
\usepackage{commons/packages/logiikka}
\usepackage{commons/packages/lukusuora}
\usepackage{commons/packages/merkkikaavio}
\usepackage{commons/packages/otsikkotyylit}
\usepackage{commons/packages/ppalkki}
\usepackage{commons/packages/qrlinkki}
\usepackage{commons/packages/termi}
\usepackage{commons/packages/todistus}
\usepackage{commons/packages/tunnisteet}
\usepackage{commons/packages/valit}


\let\cleardoublepage\clearpage

% hakemistosta section
\makeatletter
\renewenvironment{theindex}
               {\twocolumn[\vspace*{\baselineskip}\section*{Hakemisto}]%
                \@mkboth{HAKEMISTO}{HAKEMISTO}%
                \thispagestyle{plain}\parindent\z@%
                \parskip\z@ \@plus .3\p@\relax%
                \columnseprule \z@%
                \columnsep 35\p@%
                \let\item\@idxitem}%
\makeatother
