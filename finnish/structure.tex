% Kun käytät tätä, älä lataa pakettia stfloats tai fix2col. Tämä lataa
% myös paketin fixltx2e.
%% "When the new output routine for LaTeX3 is done, this package will
%% be obsolete. The sooner the better..."
\RequirePackage{dblfloatfix}

% Kun käytät tätä, et enää tarvitse paketteja type1cm ja type1ec:
\RequirePackage{fix-cm}

% Luokkamäärittelyt.
\documentclass[a4paper,onecolumn,11pt,finnish,twoside,final,oldtoc]{boek3}
\usepackage[papersize={185mm,240mm},left=36mm,right=12mm,top=10mm,bottom=25mm,%
            asymmetric,bindingoffset=10mm,marginparwidth=30mm]{geometry}
\reversemarginpar % marginaalit toisin päin
\setcounter{secnumdepth}{7} % kaikki numeroidaan ilman *-optiota

% alleviivaus yms
\usepackage[normalem]{ulem}

% cmap: "Make the PDF files generated by pdflatex "searchable and copyable"
% in Adobe (Acrobat) Reader and other compliant PDF viewers."
\usepackage{cmap}

% "Extended conditional commands"
\usepackage{xifthen}

% Erikoismerkit yms.
\usepackage{textcomp}
\usepackage{texnames}

% AMS & matematiikka
\usepackage{mathtools}
\usepackage[psamsfonts]{amssymb}
\usepackage{amsthm}
\usepackage{latexsym}

% XeLaTeX
\usepackage{fourier}
\usepackage{fontspec}
\defaultfontfeatures{Mapping=tex-text}
\setmainfont[Scale=0.92]{Heuristica}
\usepackage[protrusion=true,verbose=true]{microtype} % XeLaTeX ei toistaiseksi tue 'expansion=true'-optiota

% Otsikoiden säätelyyn
\usepackage{titlesec}

\makeatletter
\g@addto@macro\verbatim{\pdfprotrudechars=0 \pdfadjustspacing=0\relax}
\makeatother

% Babel
\usepackage[finnish,english]{babel}

% if-then-else-rakenteita helposti
\usepackage{ifthen}

% Lisää kokomäärityksiä:
\usepackage[11pt]{moresize}

% Tee jotain jokaisen sivun kohdalla (totpages tarvitsee tätä):
\usepackage{everyshi}

% Laskutehtävien suoritusta (geometry-paketti hyötyy tästä):
\usepackage{calc}

% Riviväli 1,5
\usepackage{setspace}
\onehalfspacing

% URL:ien ladonta ja "tavutus"
\usepackage[obeyspaces,spaces,hyphens,T1]{url}

% TikZ-paketti
\usepackage{pgf,tikz}
\usetikzlibrary{arrows}

% Tunnisteiden ja sivunumeroiden asettamista varten
\usepackage{fancyhdr}

% VerbatimOut Pythonia varten
\usepackage{fancyvrb}
\usepackage{fancybox}

% Euron merkki
\usepackage{eurosym}

% Hymiö
\usepackage{wasysym}

% Harjoitustehtävät answers-paketilla
\usepackage{answers}

% Cancel-paketti supistamista ym. varten
\usepackage{cancel}

% Monipalstainen sisältö esim. lyhyille tehtäville
% Käyttö: \begin{multicols}{n}
%              ...
%         \end{multicols}
\usepackage{multicol}

% Tuki numeroiduille listoille
\usepackage{enumerate}

% Desimaalipilkut
% Käyttö: väli pilkun jälkeen -> listapilkku; ei väliä pilkun jälkeen -> desimaalipilkku
\usepackage{icomma}

% Ei-kursivoidut kreikkalaiset aakkoset, esim. $\upmu$
\usepackage{upgreek}

% Automatisoitu hakemisto
\usepackage{makeidx}
\makeindex

% Aika- ja pvmformaattien kustomointiin
\usepackage{datetime}

% Täältä saa komennon \ifthispageodd
\usepackage{scrextend}

% Marginaalien muuttaminen harjoitustehtäväsivuja varten
\usepackage[strict]{changepage}

% Käännetyt tekstit taulukoita varten
\usepackage{rotating}

\usepackage{array}

% Testailuun, komentoja: \blindtext, \Blindtext, \pagevalues
\usepackage{blindtext}
\usepackage{layouts}

% Ei sisennetä kappaleen ekaa riviä
\setlength{\parindent}{0.0cm}

\theoremstyle{definition}
\newtheorem{theorem}{Teoreema}

% Lisäattribuutteja \includegraphics-komentoon
\usepackage{graphicx}

% allaolevia loogisia muuttujia voi muuttaa meta.tex-tiedostossa
% esim. \varitfalse

% väreissä vai mustavalkoisena
% vaatii vastaavat kuvatiedostot
\newif\ifvarit
\varittrue

% versionumero kansilehteen
\newif\ifversio
\versiotrue

% mikrorahoituskanavat kirjan alkuun
\newif\ifmikro
\mikrotrue

\newcommand{\kurssinTunnus}{}
\newcommand{\kurssinNimi}{}
\newcommand{\sitaatti}{Sitaatti puuttuu.}
\newcommand{\sitaatinLahde}{}
\newcommand{\kirjanVersio}{}
\newcommand{\kirjanPainopaikka}{}
\newcommand{\kirjanVastaavuus}{}
\newcommand{\metasivu}{}

% Hypersetup (metatason pdf-säätöä)
\input{hypersetup}

% Omat komennot
% pitkän matematiikan kursseja
\newcommand{\maaI}{Funktiot ja yhtälöt}
\newcommand{\maaII}{Polynomifunktiot}
\newcommand{\maaIII}{Geometria}
\newcommand{\maaIV}{Analyyttinen geometria}
\newcommand{\maaV}{Vektorit}
\newcommand{\maaVI}{Todennäköisyys ja tilastot}
\newcommand{\maaVII}{Derivaatta}
\newcommand{\maaVIII}{Juuri- ja logaritmifunktiot}
\newcommand{\maaIX}{Trigonometriset funktiot ja lukujonot}
\newcommand{\maaX}{Integraalilaskenta}
\newcommand{\maaXI}{Lukuteoria ja logiikka}
\newcommand{\maaXII}{Numeerisia ja algebrallisia menetelmiä}
\newcommand{\maaXIII}{Differentiaali- ja integraalilaskennan jatkokurssi}
\newcommand{\maaXIV}{Kertaus}

% lyhyen matematiikan kursseja
\newcommand{\mabI}{Lausekkeet ja yhtälöt}
\newcommand{\mabII}{Geometria}
\newcommand{\mabIII}{Matemaattisia malleja I}
\newcommand{\mabIV}{Matemaattinen analyysi}
\newcommand{\mabV}{Tilastot ja todennäköisyys}
\newcommand{\mabVI}{Matemaattisia malleja II}
\newcommand{\mabVII}{Talousmatematiikka}
\newcommand{\mabVIII}{Matemaattisia malleja III}
\newcommand{\mabIX}{Kertaus}

% nuolia
\newcommand{\ekvi}{\Longleftrightarrow}
\newcommand{\impli}{\Longrightarrow}

% sanoja math-moodissa
\newcommand{\kun}{\textnormal{kun} \;}
\newcommand{\jos}{\textnormal{jos} \;}
\newcommand{\tai}{\textnormal{tai} \;}

% yleisimpiä joukkoja
\newcommand{\nn}{\mathbb{N}}
\newcommand{\zz}{\mathbb{Z}}
\newcommand{\qq}{\mathbb{Q}}
\newcommand{\rr}{\mathbb{R}}
\newcommand{\cc}{\mathbb{C}}

\input{meta}

\ifvarit
    \newcommand{\vari}{}
    \definecolor{vapaa_matikka_vari_3}{cmyk}{0.2,0.27,0.0,0.0} % Laatikot
    \definecolor{vapaa_matikka_vari_4}{cmyk}{1.0,0.42,0.65,0.0} % Esimerkit
    \definecolor{vapaa_matikka_vari_5}{cmyk}{0.60,0.80,0.0,0.0} % Tehtävänumerot
\else
    \newcommand{\vari}{_bw}
    \definecolor{vapaa_matikka_vari_3}{cmyk}{0.0,0.0,0.0,0.25} % Laatikot
    \definecolor{vapaa_matikka_vari_4}{cmyk}{0.0,0.0,0.0,0.6} % Esimerkit
    \definecolor{vapaa_matikka_vari_5}{cmyk}{0.0,0.0,0.0,0.8} % Tehtävänumerot
\fi

% Tarvitaan kuvien ja taulukkojen vierekkäin laittamiseen.
\def\vcent#1{\mathsurround0pt$\vcenter{\hbox{#1}}$}

\usepackage{commons/packages/erikoissivut}
\usepackage{commons/packages/esimerkki}
\usepackage{commons/packages/kansilehti}
\usepackage{commons/packages/koosterakenteet}
\usepackage{commons/packages/kuva}
\usepackage{commons/packages/kuvaaja}
\usepackage{commons/packages/logiikka}
\usepackage{commons/packages/lukusuora}
\usepackage{commons/packages/merkkikaavio}
\usepackage{commons/packages/otsikkotyylit}
\usepackage{commons/packages/ppalkki}
\usepackage{commons/packages/qrlinkki}
\usepackage{commons/packages/termi}
\usepackage{commons/packages/todistus}
\usepackage{commons/packages/tunnisteet}
\usepackage{commons/packages/valit}


\let\cleardoublepage\clearpage

\begin{document}

\urlstyle{same}

\newdateformat{finshortdate}{\THEDAY.\THEMONTH.\THEYEAR}
\selectlanguage{finnish}
\finshortdate

\ifversio
    \kansilehti{Versio \kirjanVersio \, (\today)}
\else
    \kansilehti{}
\fi

\newpage
\subsection*{Lisenssi}

Tämän teoksen käyttöoikeutta koskee Creative Commons Nimeä 4.0 Kansainvälinen (CC BY 4.0) -lisenssi.
This work is licensed under a Creative Commons Attribution 4.0 International (CC BY 4.0) License.

\qrlinkki{http://creativecommons.org/licenses/by/4.0/deed.fi}{Lisenssin suomenkielinen tiivistelmä} \\
%\qrlinkki{http://creativecommons.org/licenses/by/4.0/deed.en}{Lisenssin englanninkielinen tiivistelmä} \\
\qrlinkki{http://creativecommons.org/licenses/by/4.0/legalcode}{Tarkat lisenssiehdot (englanniksi)}

Voit vapaasti:
\begin{description}
\item[Jakaa.] Kopioida ja jatkolevittää materiaalia, missä tahansa välineessä ja muodossa, mihin tahansa tarkoitukseen, myös kaupallisesti.
\item[Muuntaa.] Remiksata, muuntaa ja jalostaa materiaalia, mihin tahansa tarkoitukseen, myös kaupallisesti.
\end{description}

Lisenssinantaja ei voi vetää näitä vapauksia pois niin kauan kun seuraat lisenssin ehtoja.

Seuraavilla ehdoilla:
\begin{description}
\item[Nimeä.] Sinun on mainittava tekijät, näytettävä linkki lisenssiin sekä kerrottava, jos olet muuntanut materiaalia. Voit tehdä tämän millä tahansa kohtuullisella tavalla, mutta et niin, että lisenssin antaja näyttäisi tukevan sinua tai materiaalin käyttöäsi.
\item[No additional restrictions.] Et voi lisätä lakiehtoja tai teknisiä estoja, jotka estäisivät muita tekemästä mitään, minkä lisenssi sallii.
\end{description}

Huomautukset:
\begin{description}
\item[1.] Sinun ei tarvitse seurata lisenssiehtoja niiden materiaalin osien kohdalla, jotka ovat public domainissa tai joiden käyttö on sallittua tilannekohtaisen tekijänoikeuspoikkeuksen tai -rajoituksen vuoksi.
\item[2.] Mitään takuita ei ole annettu. Lisenssi ei välttämättä anna sinulle kaikkia käyttötarkoituksesi edellyttämiä oikeuksia. Esimerkiksi julkisuus-, yksityisyys- tai moraalisäädökset voivat rajoittaa materiaalin käyttöäsi.
\end{description}

\include{people}

\ifmikro
    \input{../commons/finnish/donations}
\fi

\newpage
\subsection*{Vapaa matikka}

\textbf{Git-repot} \\
\url{https://github.com/avoimet-oppimateriaalit-ry}

\textbf{Uusimmat pdf-vedokset} \\
\url{http://avoimetoppimateriaalit.fi/vapaa-matikka}

\textbf{Kirjasarjan rahoitus} \\
Teknologiateollisuuden $100$-vuotissäätiö: $30\,000$\,\euro \; (2013--2014)

\subsection*{Avoimet oppimateriaalit ry}

\textbf{Yhteystiedot} \\
\url{http://avoimetoppimateriaalit.fi} \\
\url{https://facebook.com/avoimetoppimateriaalit} \\
\url{https://facebook.com/oppikirjamaraton} \\
\href{mailto:yhdistys@avoimetoppimateriaalit.fi}{yhdistys@avoimetoppimateriaalit.fi} \\
\#avoimetoppimateriaalit \@ IRCnet

\tableofcontents
\addtocontents{toc}{\vspace*{-50pt}} % move start of toc higher on page
\input{inside} % incl. Opensolutionfile, Closesolutionfile, appendices

\newpage
\addcontentsline{toc}{chapter}{Hakemisto}
\printindex

\end{document}
